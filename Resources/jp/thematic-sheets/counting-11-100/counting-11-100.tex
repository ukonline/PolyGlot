% counting-11-100.tex
% Author: Sébastien Combéfis
% Version: August 6, 2016

\documentclass[a4paper,11pt,final]{article}

% Packages
\usepackage[utf8x]{inputenc}
\usepackage[T1]{fontenc}
\usepackage[french]{babel}
\usepackage{lmodern}

\usepackage{graphicx}
\usepackage{tikz,pgf}
\usepackage{array}
\usepackage{amssymb}
\usepackage{watermark}
\usepackage{xeCJK}
\usepackage{hyperref}
\usepackage{microtype}

% Page dimensions
\setlength\textheight{25.5cm}
\setlength\textwidth{17.5cm}
\setlength\oddsidemargin{-0.5cm}
\setlength\topmargin{-15mm}
\setlength\headheight{0mm}
\setlength\parindent{0.0cm}
\setlength\parskip{0.5cm}

% Style and fonts
\pagestyle{empty}
\urlstyle{sf}
\setmainfont{Cambria}
\setCJKmainfont{ipaexm.ttf}

% New colors
\definecolor{titleblue}{rgb}{0,0.2,0.4}
\definecolor{numberred}{rgb}{0.5,0,0}
\definecolor{sectionblue}{rgb}{0.1,0.3,0.7}
\definecolor{lightlightgray}{gray}{0.9}

% New commands
\renewcommand{\arraystretch}{1.3}
\newcommand{\trig}{\color{sectionblue}$\blacktriangleright$}
\newcommand{\sectit}[1]{\bigskip\hspace{-5mm}{\color{sectionblue}%
$\blacksquare$~~\Large\bfseries #1}}
\newcommand{\romaji}[1]{{\footnotesize[#1]}}

\begin{document}

\rightwatermark{
    \begin{tikzpicture}[overlay]
        \draw[draw=none,fill=numberred] (17,-28.7) rectangle (18,-27.2);
        \node at (17.5,-27.8) {\color{white}\thepage};
        \node[anchor=west] at (-1.5,-28) {\raisebox{-0.5mm}{%
        \includegraphics[width=2cm]{images/by-nc-nd.pdf}}%
        ~~\small poly.glot, 2016.};
    \end{tikzpicture}
}

% = = = = = = = = = = = = = = = = = = = = = = = = = = = = = = = = = = = = = = =
% Compter de 11 à 100
% = = = = = = = = = = = = = = = = = = = = = = = = = = = = = = = = = = = = = = =
\begin{tikzpicture}[overlay]
    \draw[draw=none,fill=titleblue] (4,-1) rectangle (19,1.5);
    \node[anchor=east] at (18,0.2) {\color{white}\Huge\sl Compter de 11 à 100};
    % https://openclipart.org/detail/168838/seven-segment-display-gray-7
    \node[anchor=south east] at (1.9,-0.9) {\includegraphics[width=1.2cm]%
    {images/Seven-segment-display-gray-7-200px.png}};
    \node[anchor=south east] at (3.1,-0.9) {\includegraphics[width=1.2cm]%
    {images/Seven-segment-display-gray-7-200px.png}};
\end{tikzpicture}\vspace{15mm}

% - - - - - - - - - - - - - - - - - - - - - - - - - - - - - - - - - - - - - - -

Sachant compter de 1 à 10, il est assez immédiat de compter de 11 à 100. Pour
compter de 11 à 19, on fait suivre じゅう \romaji{j\=u} du chiffre des unités :
\vspace{-5mm}
\begin{center}
	\fbox{\rule[-4mm]{0mm}{1cm}\parbox{15cm}{\centering\bf じゅう
	\romaji{j\=u} + unités}}
\end{center}

Par exemple, 11 se dit じゅういち \romaji{j\=u ichi} et 17 se dit じゅうなな
\romaji{j\=u nana} ou じゅうしち \romaji{j\=u shichi}. Pour les nombres de 20
à 99, il suffit d'ajouter le chiffre des dizaines :
\vspace{-5mm}
\begin{center}
	\fbox{\rule[-4mm]{0mm}{1cm}\parbox{15cm}{\centering\bf dizaines + じゅう
	\romaji{j\=u} (+ unités)}}
\end{center}

Par exemple, 32 se dit さんじゅうに \romaji{san j\=u ni} et 86 se dit
はちじゅうろく \romaji{hachi j\=u roku}. Notez que pour les dizaines, il est
interdit d'utiliser les lectures alternatives pour 4, 7 et 9; on doit donc
respectivement utiliser よん \romaji{yon}, なな \romaji{nana} et
きゅう \romaji{ky\=u}. Enfin, le nombre 100 se dit ひゃく \romaji{hyaku}.

% - - - - - - - - - - - - - - - - - - - - - - - - - - - - - - - - - - - - - - -

\sectit{Les dizaines}

\hspace{5mm}\begin{tabular}{|p{1.5cm}p{1.5cm}p{4cm}}
    10    & 十    & じゅう \romaji{j\=u} \\
    20    & 二十    & にじゅう \romaji{ni j\=u} \\
    30    & 三十    & さんじゅう \romaji{san j\=u} \\
    40    & 四十    & よんじゅう \romaji{yon j\=u} \\
    50    & 五十    & ごじゅう \romaji{go j\=u} \\
    60    & 六十    & ろくじゅう \romaji{roku j\=u} \\
    70    & 七十    & ななじゅう \romaji{nana j\=u} \\
    80    & 八十    & はちじゅう \romaji{hachi j\=u} \\
    90    & 九十    & きゅうじゅう \romaji{ky\=u j\=u} \\
    100   & 百    & ひゃく \romaji{hyaku}
\end{tabular}

% - - - - - - - - - - - - - - - - - - - - - - - - - - - - - - - - - - - - - - -

\sectit{Écriture}

Pour des documents officiels légaux ou financiers, il existe également un kanji
complexe pour 100, pour en empêcher la falsification. Il n'est cependant que
peu utilisé en japonais, 百 étant suffisamment complexe.

\hspace{5mm}\begin{tabular}{|l*{1}{p{1cm}}}
    \multicolumn{1}{l}{} & 100 \\
    \it\small Écriture commune & 百 \\
    \it\small Écriture complexe & 佰
\end{tabular}

\end{document}