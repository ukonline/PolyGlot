% other-family-close-members.tex
% Author: Sébastien Combéfis
% Version: August 6, 2016

\documentclass[a4paper,11pt,final]{article}

% Packages
\usepackage[utf8x]{inputenc}
\usepackage[T1]{fontenc}
\usepackage[french]{babel}
\usepackage{lmodern}

\usepackage{graphicx}
\usepackage{tikz,pgf}
\usepackage{array}
\usepackage{amssymb}
\usepackage{watermark}
\usepackage{xeCJK}
\usepackage{wasysym}
\usepackage{hyperref}
\usepackage{microtype}

% Page dimensions
\setlength\textheight{25.5cm}
\setlength\textwidth{17.5cm}
\setlength\oddsidemargin{-0.5cm}
\setlength\topmargin{-15mm}
\setlength\headheight{0mm}
\setlength\parindent{0.0cm}
\setlength\parskip{0.5cm}

% Style and fonts
\pagestyle{empty}
\urlstyle{sf}
\setmainfont{Cambria}
\setCJKmainfont{ipaexm.ttf}

% New colors
\definecolor{titleblue}{rgb}{0,0.2,0.4}
\definecolor{numberred}{rgb}{0.5,0,0}
\definecolor{sectionblue}{rgb}{0.1,0.3,0.7}
\definecolor{lightlightgray}{gray}{0.9}

% New commands
\renewcommand{\arraystretch}{1.3}
\newcommand{\trig}{\color{sectionblue}$\blacktriangleright$}
\newcommand{\sectit}[1]{\bigskip\hspace{-5mm}{\color{sectionblue}$\blacksquare$~~\Large\bfseries #1}}
\newcommand{\romaji}[1]{{\footnotesize[#1]}}

\begin{document}

\rightwatermark{
    \begin{tikzpicture}[overlay]
        \draw[draw=none,fill=numberred] (17,-28.7) rectangle (18,-27.2);
        \node at (17.5,-27.8) {\color{white}\thepage};
        \node[anchor=west] at (-1.5,-28) {\raisebox{-0.5mm}{%
        \includegraphics[width=2cm]{images/by-nc-nd.pdf}}%
        ~~\small poly.glot, 2016.};
    \end{tikzpicture}
}

% = = = = = = = = = = = = = = = = = = = = = = = = = = = = = = = = = = = = = = =
% Membres proches d'une autre famille
% = = = = = = = = = = = = = = = = = = = = = = = = = = = = = = = = = = = = = = =
\begin{tikzpicture}[overlay]
    \draw[draw=none,fill=titleblue] (4,-1) rectangle (19,1.5);
    \node[anchor=east] at (18,0.2) {\color{white}\Huge\sl Membres proches d'une
    autre famille};
    % https://openclipart.org/detail/188816/family-bw-version
    \node[anchor=south east] at (3,-1.15) {\includegraphics[width=3cm]%
	{images/bw-family-400px.png}};
\end{tikzpicture}\vspace{15mm}

% - - - - - - - - - - - - - - - - - - - - - - - - - - - - - - - - - - - - - - -

Il y a deux façons de s'adresser aux membres d'une famille en japonais, selon
qu'il s'agisse de parler des membres de sa famille à d'autres personnes ou
de parler des membres de la famille d'une autre personne. Famille se dit
家族/かぞく \romaji{kazoku} en japonais; ce terme voulant également dire
\textit{\og membres de la famille \fg}. Lorsqu'on parle des membres d'une autre
famille, on utilisera le mot formel ご家族/ごかぞく \romaji{gokazoku}.

% - - - - - - - - - - - - - - - - - - - - - - - - - - - - - - - - - - - - - - -

\sectit{Membres proches}

Le diagramme suivant reprend les termes utilisés pour parler des \emph{membres
proches d'une famille d'une autre personne}. On remarquera qu'il y a une
nuance, en japonais, entre les grands et petits frères et s\oe urs.

\begin{center}
\scalebox{0.9}{\begin{tikzpicture}[box/.style={draw,shape=rectangle}]
    \useasboundingbox (-3,-0.75) rectangle (3.5,3.5);
    \node[box] (me) at (0,0) {私};
    \node[box] (older-bro) at (-2,0.5) {お兄さん};
    \node[box] (younger-bro) at (-2,-0.5) {弟さん};
    \node[box] (older-sis) at (2,0.5) {お姉さん};
    \node[box] (younger-sis) at (2,-0.5) {妹さん};
    \node[box] (father) at (-1.25,2) {お父さん};
    \node[box] (mother) at (1.25,2) {お母さん};
    \node[box] (grand-father) at (0,3) {お爺さん};
    \node[box] (grand-mother) at (2.5,3) {お婆さん};
    %%%
    \draw (father) -- (-1.25,1.5) -- (1.25,1.5) -- (mother);
    \draw (0,1.5) -- (me);
    \draw (older-bro) -- (-2,1) -- (2,1) -- (older-sis);
    \draw (older-bro) -- (younger-bro);
    \draw (older-sis) -- (younger-sis);
    \draw (grand-father) -- (0,2.5) -- (2.5,2.5) -- (grand-mother);
    \draw (1.25,2.5) -- (mother);
    %%%
    \node at (-2,-1.25) {\mars};
    \node at (2,-1.25) {\female};
\end{tikzpicture}}
\end{center}

On remarquera immédiatement l'utilisation systématique du suffixe honorifique
さん \romaji{san} pour désigner les membres de la famille.

Notez également que vous pouvez utiliser les mots pour grand-père et
grand-mère pour vous adresser à des personnes âgées inconnues qui ont l'âge
de vos grand-parents.

% - - - - - - - - - - - - - - - - - - - - - - - - - - - - - - - - - - - - - - -

\sectit{Vocabulaire}

\hspace{5mm}\begin{tabular}{|p{2cm}p{4.5cm}l}
    家族     & かぞく \romaji{kazoku}            & Famille, membres de la famille \\
    お爺さん     & おじいさん \romaji{ojiisan}   & Grand-père \\
    お婆さん  & おばあさん \romaji{obaasan}  & Grand-mère \\
    両親     & りょうしん \romaji{ry\=oshin} & Parents \\
    お父さん  & おとうさん \romaji{ot\=osan} & Père \\
    お母さん  & おかあさん \romaji{okaasan}  & Mère \\
    お兄さん  & おにいさん \romaji{oniisan}  & Grand frère \\
    お姉さん  & おねえさん \romaji{oneesan}  & Grande s\oe ur \\
    弟さん  & おとうとさん \romaji{ot\=otosan}  & Petit frère \\
    妹さん  & いもうとさん \romaji{im\=otosan}  & Petite s\oe ur
\end{tabular}

\end{document}