% saying-goodbye.tex
% Author: Sébastien Combéfis
% Version: June 20, 2016

\documentclass[a4paper,11pt,final]{article}

% Packages
\usepackage[utf8x]{inputenc}
\usepackage[T1]{fontenc}
\usepackage[french]{babel}
\usepackage{lmodern}

\usepackage{graphicx}
\usepackage{tikz,pgf}
\usepackage{array}
\usepackage{amssymb}
\usepackage{watermark}
\usepackage{xeCJK}
\usepackage{hyperref}
\usepackage{microtype}

% Page dimensions
\setlength\textheight{25.5cm}
\setlength\textwidth{17.5cm}
\setlength\oddsidemargin{-0.5cm}
\setlength\topmargin{-15mm}
\setlength\headheight{0mm}
\setlength\parindent{0.0cm}
\setlength\parskip{0.5cm}

% Style and fonts
\pagestyle{empty}
\urlstyle{sf}
\setmainfont{Cambria}
\setCJKmainfont{ipaexm.ttf}

% New colors
\definecolor{titleblue}{rgb}{0,0.2,0.4}
\definecolor{numberred}{rgb}{0.5,0,0}
\definecolor{sectionblue}{rgb}{0.1,0.3,0.7}
\definecolor{lightlightgray}{gray}{0.9}

% New commands
\renewcommand{\arraystretch}{1.3}
\newcommand{\trig}{\color{sectionblue}$\blacktriangleright$}
\newcommand{\sectit}[1]{\bigskip\hspace{-5mm}{\color{sectionblue}$\blacksquare$~~\Large\bfseries #1}}
\newcommand{\romaji}[1]{{\footnotesize[#1]}}

\begin{document}

\rightwatermark{
    \begin{tikzpicture}[overlay]
        \draw[draw=none,fill=numberred] (17,-28.7) rectangle (18,-27.2);
        \node at (17.5,-27.8) {\color{white}\thepage};
        \node[anchor=west] at (-1.5,-28) {\raisebox{-0.5mm}{%
        \includegraphics[width=2cm]{images/by-nc-nd.pdf}}%
        ~~\small poly.glot, 2016.};
    \end{tikzpicture}
}

% = = = = = = = = = = = = = = = = = = = = = = = = = = = = = = = = = = = = = = =
% Dire au revoir
% = = = = = = = = = = = = = = = = = = = = = = = = = = = = = = = = = = = = = = =
\begin{tikzpicture}[overlay]
    \draw[draw=none,fill=titleblue] (4,-1) rectangle (19,1.5);
    \node[anchor=east] at (18,0.2) {\color{white}\Huge\sl Dire au revoir};
    % https://openclipart.org/detail/21857/comic-characters-bye
    \node[anchor=south east] at (3,-1.15) {\includegraphics[width=3cm]%
    {images/nicubunu-Comic-characters-Bye-200px.png}};
\end{tikzpicture}\vspace{15mm}

% - - - - - - - - - - - - - - - - - - - - - - - - - - - - - - - - - - - - - - -

Il y a plusieurs façons de dire \textit{au revoir} en japonais, qui dépendent
du type d'au revoir et de la personne à qui on s'adresse. De manière générique,
on utilise さようなら \romaji{say\=onara} pour dire au revoir.

% - - - - - - - - - - - - - - - - - - - - - - - - - - - - - - - - - - - - - - -

\sectit{Selon la situation}

La première distinction qu'on peut faire concerne le type d'au revoir que l'on
fait. Cette nuance ainsi apportée est importante au Japon, car elle apporte
plus d'informations qu'un simple au revoir.

\hspace{5mm}\begin{tabular}{|p{3.25cm}p{5cm}l}
    \multicolumn{1}{l}{}&& \it\small Utilisation \\
    左様なら              & さようなら \romaji{say\=onara}            & Peu
    utilisé (générique) \\
    行って来ます        & いってきます \romaji{itte kimasu}        & En
    quittant la maison (celui qui quitte) \\
    行ってらっしゃい  & いってらっしゃい \romaji{itte rasshai} & En quittant la
    maison (celui qui reste) \\
    また明日              & またあした \romaji{mata ashita}           & À
    demain \\
    また来週              & またらいしゅう \romaji{mata raish\=u}   & À la
    semaine prochaine \\
    お休みなさい        & おやすみなさい \romaji{oyasumi nasai}   & Bonne nuit
\end{tabular}

On notera qu'on peut mettre en évident le moment où l'on reverra la personne
qu'on quitte en commençant avec また \romaji{mata} qu'on peut traduire par
\og\textit{À...}\fg. Enfin, on utilise おやすみなさい \romaji{oyasumi nasai}
lorsqu'on dit au revoir à quelqu'un à l'heure où on va usuellement dormir.
Lorsqu'on quitte quelqu'un pour une plus longue durée et qu'on veut lui
souhaiter de prendre soin de lui, il existe une série d'autres formules.

\hspace{5mm}\begin{tabular}{|p{2.5cm}p{5cm}l}
    \multicolumn{1}{l}{}&& \it\small Utilisation \\
    気をつけて  & きをつけて \romaji{ki wo tsukete}    & Quelqu'un qui quitte
    la maison, part en vacances  \\
    元気で        & げんきで \romaji{genki de}            & Quelqu'un qui part
    en voyage ou déménage loin \\
    お大事に     & おだいじに \romaji{odaiji ni}        & Quelqu'un qui est
    malade
\end{tabular}

Les deux premières expressions peuvent se traduire par \og\textit{Prends soin
de toi.}\fg{}, \og\textit{Je te souhaite le meilleur.}\fg{} et la dernière par
\og\textit{Bon rétablissement.}\fg{}.

% - - - - - - - - - - - - - - - - - - - - - - - - - - - - - - - - - - - - - - -

\sectit{Discours informel}

\hspace{5mm}\begin{tabular}{|p{2cm}p{4.5cm}l}
    \multicolumn{1}{l}{}&& \it\small Utilisation \\
    & じゃあね \romaji{j\=a ne}    & Entre amis (familier) \\
    & またね \romaji{mata ne}       & Entre amis (familier) \\
    & バイバイ \romaji{bai bai}    & Jeunes filles (familier)
\end{tabular}

Entre amis, il est commun d'utiliser じゃあね \romaji{j\=a ne}, voire バイバイ
\romaji{bai bai} entre filles qu'on peut traduire par \og\textit{Ciao.}\fg{} ou
\og\textit{Bye bye.}\fg{} De plus, on peut se limiter à じゃあね \romaji{j\=a}
et またね \romaji{mata} avec des proches. On peut également utiliser またね
\romaji{mata ne} qu'on peut traduire par \og\textit{À plus (tard).}\fg{}

\end{document}