\documentclass[a4paper,11pt,final]{article}

% Packages
\usepackage[utf8x]{inputenc}
\usepackage[T1]{fontenc}
\usepackage[french]{babel}
\usepackage{lmodern}

\usepackage{graphicx}
\usepackage{tikz,pgf}
	\usetikzlibrary{arrows,decorations.pathreplacing}
\usepackage{array}
\usepackage{amssymb}
\usepackage{watermark}
\usepackage{xeCJK}
\usepackage{hyperref}
\usepackage{microtype}

% Page dimensions
\setlength\textheight{25.5cm}
\setlength\textwidth{17.5cm}
\setlength\oddsidemargin{-0.5cm}
\setlength\topmargin{-15mm}
\setlength\headheight{0mm}
\setlength\parindent{0.0cm}
\setlength\parskip{0.5cm}

% Style and fonts
\pagestyle{empty}
\urlstyle{sf}
\setmainfont{Cambria}
\setCJKmainfont{ipaexm.ttf}

% New colors
\definecolor{titleblue}{rgb}{0,0.2,0.4}
\definecolor{numberred}{rgb}{0.5,0,0}
\definecolor{sectionblue}{rgb}{0.1,0.3,0.7}
\definecolor{lightlightgray}{gray}{0.9}

% New commands
\renewcommand{\arraystretch}{1.3}
\newcommand{\trig}{\color{sectionblue}$\blacktriangleright$}
\newcommand{\sectit}[1]{\bigskip\hspace{-5mm}{\color{sectionblue}$\blacksquare$~~\Large\bfseries #1}}
\newcommand{\romaji}[1]{{\footnotesize[#1]}}

\begin{document}
	\rightwatermark{
		\begin{tikzpicture}[overlay]
			\draw[draw=none,fill=numberred] (17,-28.7) rectangle (18,-27.2);
			\node at (17.5,-27.8) {\color{white}\thepage};
			\node[anchor=west] at (-1.5,-28) {\raisebox{-0.5mm}{\includegraphics[width=2cm]{images/by-nc-nd.pdf}}~~\small poly.glot, 2016.};
		\end{tikzpicture}
	}

% = = = = = = = = = = = = = = = = = = = = = = = = = = = = = = = = = = = = = = = = = = = = = = = = = = = =
% Dire bonjour
% = = = = = = = = = = = = = = = = = = = = = = = = = = = = = = = = = = = = = = = = = = = = = = = = = = = =
\begin{tikzpicture}[overlay]
	\draw[draw=none,fill=titleblue] (4,-1) rectangle (19,1.5);
	\node[anchor=east] at (18,0.2) {\color{white}\Huge\sl Dire bonjour};
	% https://openclipart.org/detail/247271/man-hello
	\node[anchor=south east] at (2.5,-1.15) {\includegraphics[width=1.5cm]{images/hombre-hello-remix-cyberscooty-200px.png}};
\end{tikzpicture}\vspace{15mm}
	
% - - - - - - - - - - - - - - - - - - - - - - - - - - - - - - - - - - - - - - - - - - - - - - - - - - - -

Il y a plusieurs façons de dire \textit{bonjour} en japonais, qui dépendent notamment du moment de la journée et de la personne à qui on s'adresse. De manière générique, on utilise こんにちは \romaji{konnichiwa} pour dire bonjour.

% - - - - - - - - - - - - - - - - - - - - - - - - - - - - - - - - - - - - - - - - - - - - - - - - - - - -

\sectit{Au long de la journée}

La première distinction qu'on peut faire concerne le moment de la journée où on rencontre pour la première fois la personne à qui on dit bonjour. Cette notion du moment de la journée est importante au Japon, tout comme le fait qu'il faille s'incliner lorsqu'on salue quelqu'un.

\hspace{5mm}\begin{tabular}{|p{2cm}p{4.5cm}l}
	\multicolumn{1}{l}{}&& \it\small Utilisation \\
	お早う			& おはよう \romaji{ohay\=o}				& Dans la matinée (avant le lunch) \\
	今日は			& こんにちは \romaji{konnichiwa}			& Durant l'après-midi (générique) \\
	今晩は			& こんばんは \romaji{konbanwa}			& Dans la soirée (après le souper)
\end{tabular}

On notera qu'on utilisera sans soucis こんにちは \romaji{konnichiwa} entre le lunch (pas avant) et le souper (pas après). De plus, on n'utilise おはよう \romaji{ohay\=o} qu'avec des amis ou proches, la version polie étant おはよう ございます \romaji{ohay\=o gozaimasu}. Enfin, lorsqu'on dit bonjour à quelqu'un au téléphone, c'est-à-dire au moment où on décroche, on utilisera plutôt もしもし \romaji{mochimochi}, utilisé donc par la personne appelée.

\begin{center}
\scalebox{0.9}{\begin{tikzpicture}
	\useasboundingbox (-0.2,-1.2) rectangle (11.2,1.5);
	\draw[-angle 60,line width=0.5mm] (0,0) -- (11,0);
	\draw[line width=0.5mm] (2,-0.2) -- (2,0.2);
	\node[anchor=south] at (2,0.2) {\small 12h};
	\draw[line width=0.5mm] (3,-0.2) -- (3,0.2);
	\node[anchor=south] at (3,0.2) {\small 13h};
	\draw[line width=0.5mm] (8,-0.2) -- (8,0.2);
	\node[anchor=south] at (8,0.2) {\small 18h};
	\draw[line width=0.5mm] (9,-0.2) -- (9,0.2);
	\node[anchor=south] at (9,0.2) {\small 19h};
	%%%
	\draw (3,-0.5) edge[decorate,decoration=brace] node[below=2mm] {おはよう ございます} (0,-0.5);
	\draw (2,0.8) edge[decorate,decoration=brace] node[above=2mm] {こんにちは} (9,0.8);
	\draw (10.95,-0.5) edge[decorate,decoration=brace] node[below=2mm] {こんばんは} (8,-0.5);
\end{tikzpicture}}
\end{center}

% - - - - - - - - - - - - - - - - - - - - - - - - - - - - - - - - - - - - - - - - - - - - - - - - - - - -

\sectit{Discours informel}

\hspace{5mm}\begin{tabular}{|p{2cm}p{4.5cm}l}
	\multicolumn{1}{l}{}&& \it\small Utilisation \\
					& やあ \romaji{y\=a} / よう \romaji{y\=o}	& Entre amis (familier) \\
	押忍				& おっす \romaji{ossu}					& Entre amis proches (homme) \\
	最近どう?		& さいきんどう? \romaji{saikin d\=o?}		& Proche ou collaborateur \\
	久しぶり			& ひさしぶり \romaji{hisashiburi}			& Ami ou proche pas revu depuis longtemps
\end{tabular}

Entre amis, il est commun d'utiliser やあ \romaji{y\=a} ou よう \romaji{y\=o}, voire même de ne même pas se saluer et directement entamer la conversation. Les deux derniers mots présentés ci-dessus seront parfois utilisé après avoir dit simplement bonjour, bien qu'on puisse les utiliser seul. On peut traduire さいきんどう? \romaji{saikin d\=o?} par \og\textit{Quelles nouvelles ?}\fg{} et ひさしぶり \romaji{hisashiburi} par \og\textit{Ça fait longtemps qu'on ne s'est pas vus!}\fg.

\end{document}